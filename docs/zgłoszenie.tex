% !TEX program = xelatex
%Wzór dokumentu
%tu zmień marginesy i rozmiar czcionki
\documentclass[a4paper,11pt]{article}
\usepackage{inputenc}[utf8]
\usepackage[margin=2.8cm]{geometry}
\usepackage[polish]{babel}

%Lepiej tego nie zmieniaj, jak co to dodawaj pakiety
\usepackage{titlesec}
\usepackage{titling}
\usepackage{fancyhdr}
\usepackage{mdframed}
\usepackage{graphicx}
\usepackage{amsmath}
\usepackage{amsfonts}
\usepackage{multicol}
\usepackage{multirow}
\usepackage{listings}
\usepackage{caption}
\usepackage{float}
\usepackage{pdfpages}
\usepackage{tikz}
	\usetikzlibrary{arrows}
	\usetikzlibrary{patterns}
	\usetikzlibrary{decorations.pathmorphing}
\usepackage{pgf}
\usepackage[section]{placeins}



%inny wygląd
%\usepackage{tgbonum}


\usepackage{hyperref}
\hypersetup{
    colorlinks=true,
    linkcolor=blue,
    filecolor=magenta,      
    urlcolor=cyan,
}

\urlstyle{same}
%Zmienne, zmień je!
\graphicspath{ {./ilustracje/} }
\title{Wyznaczanie zależności zasięgu strumienia wodyod ciśnienia hydrostatycznego}
\author{Grzegorz Koperwas}
\date{\today}

%lokalizacja polska (odkomentuj jak piszesz po polsku)

\usepackage{polski}
\usepackage[polish]{babel} 
\usepackage{indentfirst}
\usepackage{icomma} 

\brokenpenalty=1000
\clubpenalty=1000
\widowpenalty=1000    

%nie odkometowuj wszystkiego, użyj mózgu
%\renewcommand\thechapter{\arabic{chapter}.}
\renewcommand\thesection{\arabic{section}.}
\renewcommand\thesubsection{\arabic{section}.\arabic{subsection}.}
\renewcommand\thesubsubsection{\arabic{subsubsection}.}

%Makra

\newcommand{\obrazek}[2]{
\begin{figure}[h]
    \centering
    \includegraphics[scale=#1]{#2}
\end{figure}
}     

\newcommand{\stopnie}{\ensuremath{^{\circ}}}

\newcommand{\twierdzonko}[1]{
    \begin{center}
    \begin{mdframed}
    #1
    \end{mdframed}          
    \end{center}
} 

\newcommand{\dwanajeden}[2]{
\ensuremath \left( \begin{array}{c}
    #1\\
    #2
\end{array} \right)
}  

%Stopka i head (sekcja której nie powinno się zmieniać)
\pagestyle{fancy}
\fancyhead{}
\fancyfoot{}

%Zmieniaj od tego miejsca
\rfoot{}
\lfoot{}
\lhead{}
\rhead{}
\renewcommand{\headrulewidth}{0pt}
\renewcommand{\footrulewidth}{0pt}



\begin{document}
\begin{center}
    \begin{Large}
        Formularz zgłoszenia projektu
    \end{Large}
\end{center}

\subsubsection*{Nazwisko i Imię członka zespołu oraz numer sekcji (wg listy studentów)}

\begin{enumerate}
    \item Koperwas Grzegorz; sekcja 2 
    \item Patryk Sroczyński; [REZYGNACJA]
\end{enumerate}

\subsubsection*{Temat Projektu}

Gra gatunku \emph{Rouge}\footnote{Losowo generowany świat, śmierć oznacza nowy początek i nowy świat.}, mająca raczej charakter dema technicznego.

\subsubsection*{Krótki opis projektu (założenia projektu, w tym informacja o wykorzystanym języku, środowisku i bibliotekach).}

Celem projektu jest stworzenie frameworka zapewniającego abstrakcję niskopoziomowych bibliotek, oraz zapewniającego system Entity-Component.

\vspace{5mm}

\textbf{Języki programowania:}
\begin{itemize}
    \item C++ (Standard 17 lub 20), raczej z rozszerzeniami GNU.
    \item GLSL, do pisania shaderów
    \item Bash, skrypty buildowania zależności
    \item Inne języki skryptowe w ramach potrzeby (tylko buildtime)
\end{itemize}

\textbf{Środowisko} - Nie będzie wykorzystywane środowisko pełne, tylko system buildowania aplikacji \href{https://mesonbuild.com/}{Meson}, wraz z backendem \href{https://ninja-build.org/}{Ninja} i kompilatorem gcc. Zostanie dostarczony również plik \texttt{dockerfile} do kompilacji krzyżowej na windowsa.

Jako edytor będzie używany \texttt{nvim}\footnote{Wersja nightly, czyli kompilowana z mastera}. Z innych ciekawych programów to debugger wycieków pamięci \texttt{valgrid}. Dokumentacja jest pisana z użyciem programu \LaTeX{} oraz jako edytor jest używane \texttt{Visual Studio Code}.


\vspace{5mm}

\textbf{Biblioteki:}
\begin{itemize}
    \item SFML - obsługa okien i abstrakcja OpenGL
    \item \href{https://box2d.org/}{box2d} - Fizyka 2D
    \item \href{https://github.com/aaronmjacobs/Boxer}{boxer} - Popupy do wyświetlania błędów, takich jak brak obsługi Shaderów
    \begin{itemize}
        \item GTK+ 3 - do obsługi popupów na Linuxie
    \end{itemize}
    \item OpenAL - Odtwarzanie audio przestrzennego
    \item libavcodec, libavdevice, libavfilter, libavformat, libavutil, libpostproc, libswresample, libswscale (ffmpeg) - dekompresja i filtry audio; może tekstura video?
    \item pthread/win32pthread lub inna biblioteka do obsługi wątków.
\end{itemize}


Projekt jest jako repozytorium Git'a dostępny pod adresem \url{https://github.com/HakierGrzonzo/GrzesSFMLlib}.\footnote{Z chwilą zgłaszania zostało napisane 1200 linijek.}

\end{document}
