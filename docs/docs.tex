% !TEX program = xelatex
%Wzór dokumentu
%tu zmień marginesy i rozmiar czcionki
\documentclass[a4paper,11pt]{article}
\usepackage{inputenc}[utf8]
\usepackage[margin=2.5cm]{geometry}
\usepackage[polish]{babel}

%Lepiej tego nie zmieniaj, jak co to dodawaj pakiety
\usepackage{titlesec}
\usepackage{titling}
\usepackage{fancyhdr}
\usepackage{mdframed}
\usepackage{graphicx}
\usepackage{amsmath}
\usepackage{amsfonts}
\usepackage{multicol}
\usepackage{multirow}
\usepackage{listings}
\usepackage{caption}
\usepackage{float}
\usepackage{pdfpages}
\usepackage{tikz}
	\usetikzlibrary{arrows}
	\usetikzlibrary{patterns}
	\usetikzlibrary{decorations.pathmorphing}
\usepackage{pgf}
\usepackage[section]{placeins}



%inny wygląd
%\usepackage{tgbonum}


\usepackage{hyperref}
\hypersetup{
    colorlinks=true,
    linkcolor=blue,
    filecolor=magenta,      
    urlcolor=cyan,
}

\urlstyle{same}
%Zmienne, zmień je!
\graphicspath{ {./ilustracje/} }
\title{Dokumentacja projektu GrzesSFMLlib}
\author{Grzegorz Koperwas}
\date{\today}

%lokalizacja polska (odkomentuj jak piszesz po polsku)

\usepackage{polski}
\usepackage[polish]{babel} 
\usepackage{indentfirst}
\usepackage{icomma} 

\brokenpenalty=1000
\clubpenalty=1000
\widowpenalty=1000    

%nie odkometowuj wszystkiego, użyj mózgu
%\renewcommand\thechapter{\arabic{chapter}.}
\renewcommand\thesection{\arabic{section}.}
\renewcommand\thesubsection{\arabic{section}.\arabic{subsection}.}
\renewcommand\thesubsubsection{\arabic{subsubsection}.}

%Makra

\newcommand{\obrazek}[3]{
\begin{figure}[p]
    \centering
    \includegraphics[scale=#1]{#2}
    \caption{#3}\label{obr:#2}
\end{figure}
}     

\newcommand{\stopnie}{\ensuremath{^{\circ}}}

\newcommand{\twierdzonko}[1]{
    \begin{center}
    \begin{mdframed}
    #1
    \end{mdframed}          
    \end{center}
} 

\newcommand{\dwanajeden}[2]{
\ensuremath \left( \begin{array}{c}
    #1\\
    #2
\end{array} \right)
}  

%Stopka i head (sekcja której nie powinno się zmieniać)
\pagestyle{fancy}
\fancyhead{}
\fancyfoot{}

%Zmieniaj od tego miejsca
\rfoot{\thepage}
\lfoot{}
\rhead{}
\lhead{Ostatnia edycja: \today}
\renewcommand{\headrulewidth}{1pt}
\renewcommand{\footrulewidth}{1pt}



\begin{document}

\thispagestyle{empty}
\begin{minipage}{\textwidth}
\vspace{2cm}
\begin{flushright}
    \LARGE{\textbf{\thetitle}}

    \vspace{1cm}

    \Large{\theauthor}

    \makeatletter
    \@date
    \makeatother
\end{flushright}
\end{minipage}

\vspace{4cm}

\tableofcontents

\section{Instrukcje dotyczące kompilacji}

Projekt został wykonany w środowisku \texttt{meson}, zalecane jest użycie systemu linux i kompilacji krzyżowej w celu kompilacji projektu.

Projekt był testowany jedynie w architekturze x86\_64, i686\footnote{32 bitowe pentium III+} nie jest wspierane.


\subsection*{Potrzebne biblioteki i inne zalecane programy:}

\begin{itemize}
    \item Wymagane programy:
    \begin{itemize}
        \item gcc - kompilator
        \item meson - środowisko budujące
        \item GNU make - automatyzacja budowania i uruchamiania.
    \end{itemize}
    \item Wymagane biblioteki:
    \begin{itemize}
        \item sfml
        \item box2d
        \item OpenAL[-soft]
        \item libavcodec, libavutil, libavformat, libswresample - powinny być zainstalowane wraz z \texttt{ffmpeg}'iem, skompilowane biblioteki powinny mieć włączony kodek \texttt{aac}.
        \item Dla systemu linux:
        \begin{itemize}
            \item GTK+ 3
            \item \href{https://github.com/aaronmjacobs/Boxer}{boxer} - należy zainstalować bibliotekę samemu do \texttt{/lib64/libBoxer.a}
        \end{itemize}
        Dla systemu windows obsługa popupów z błędami została wyłączona z powodu trudności z znalezieniem pliku \texttt{windows.h}, obsługa popupów na systemie Windows jest możliwa, jednak nie została uznana za konieczną.
    \end{itemize}
\end{itemize}

\subsection*{Kompilacja krzyżowa}

Kompilację krzyżową można zrealizować za pomocą narzędzi \texttt{mingw-w64-*}, jednak trzeba wtedy samodzielnie skompilować wszystkie biblioteki za pomocą tych narzędzi. \emph{Arch Linux} w repozytoriach społeczności posiada wersje wielu z tych programów i bibliotek pod mingw właśnie. Projekt posiada skrypt \texttt{makewinBuild.sh} który kopiuje wybudowany plik wykonywalny, assety oraz pliki \texttt{dll} do folderu \texttt{release/} i go zipuje. Taka konfiguracja może łatwo być testowana za pomocą środowiska \texttt{WINE}\footnote{WINE - ,,Wine Is Not An Emulator'' - środowisko zastępujące biblioteki windowsowe unixowymi, pozwalające programom windowsowym działać na systemach unixopodobnych}.

\subsection*{Obsługa projektu (GNU/Linux)}

Najpierw należy wykonać polecenie
\begin{lstlisting}[frame=single,basicstyle=\ttfamily]
$ make setup
\end{lstlisting}
by meson zainicjalizował pliki projektu. Meson sprawdzi wtedy obecność wszystkich bibliotek do kompilacji normalnej i krzyżowej.

Projekt kompilujemy poleceniem:
\begin{lstlisting}[frame=single,basicstyle=\ttfamily]
$ make build
\end{lstlisting}
lub dla systemu windows:
\begin{lstlisting}[frame=single,basicstyle=\ttfamily]
$ make buildWindows
\end{lstlisting}

Pliki wykonywalne znajdą się w folderach \texttt{Build/} oraz \texttt{mingw/}. By uruchomić build linuxowy możemy wydać polecenie:
\begin{lstlisting}[frame=single,basicstyle=\ttfamily]
$ make run
\end{lstlisting}

Jeżeli wystąpiły jakieś zmiany to meson je zbuduje.

\subsection*{Obsługa projektu (Windows)}

Zainstaluj WSL. Narzędzia mingw64, mingw32 oraz msys2 nie dadzą rady. Adaptowanie projektu pod visual'a na własną odpowiedzialność.

\section{Instrukcja obsługi}

\subsection*{Wymagania minimalne}

\begin{itemize}
    \item Karta graficzna z obsługą OpenGL 3.0 lub lepiej
    \item 64 bitowy system GNU/Linux lub 64 bitowy system Windows
    \item Klawiatura, mysz
\end{itemize}

\subsection*{Sterowanie}

\begin{itemize}
    \item Poruszanie klawiszami \texttt{WASD}
    \item Strzelanie spacją
    \item Spowolnienie czasu - lewy ctrl
    \item Celowanie myszą
\end{itemize}

Sterowanie jest również objaśnione na obrazku \ref{obr:controls.png}, wyświetlane jest na początku rozgrywki.

Po zakończeniu gry, które jest zapowiedziane coraz bardziej czerwonym ekranem (patrz obrazek \ref{obr:lowHP.png}), wyświetlony zostanie czas gry (z uwzględnieniem spowolnienia czasu).

Przeciwnicy wybuchają kiedy znajdą się blisko gracza, oraz jako iż starają się trzymać w kupie to dochodzi do widowiskowych reakcji łańcuchowych.

Jeżeli okno gry straci focus, to rozgrywka zostanie spowolniona znacząco.

\obrazek{.2}{controls.png}{Początek rozgrywki}

\obrazek{.2}{enemies.png}{Przeciwnicy starający się atakować w grupie}

\obrazek{.2}{lowHP.png}{Prawie że koniec rozgrywki}

\subsection*{Różnice między wersją Windows a GNU/Linux}

Wersja na systemy windows nie posiada następującej funkcjonalności:
\begin{itemize}
    \item Popup'y z błędami.
    \item Tryb \emph{niskiej wydajności} (Brak anty aliasingu, limit klatek do
        30 na sekundę) jeżeli program został uruchomiony na laptopie który
        działa na baterii\footnote{funkcjonalność powstała by oszczędzać baterię
        podczas pracy nad projektem, kiedy za oknem panowie elektrycy dzielnie
        wymieniali słupy.}.
\end{itemize}

\section{Dokumentacja techniczna}

\subsection{Założenia oraz filozofia aplikacji}

Celem projektu było początkowo stworzenie gry, ale w jego trakcie okazało się że zabawa w implementację wielu różnych funkcjonalności jest o wiele ciekawsze niż pisanie gry, która te funkcjonalności by wykorzystywała. Mimo podczas pisania projektu kierowałem się następującymi zasadami:

\begin{enumerate}
    \item \textbf{Sprytne wskaźniki zamiast ,,gołych'' wskaźników} - smart pointer'y okazały się bardzo pomocną rzeczą, zastosowanie klasy \texttt{std::shared\_ptr} pozwoliło na wyeliminowanie problemów z żywotnością obiektów. Obiekty klasy \texttt{entity::EntitySystem} posiadją wskaźniki dzielone, a obiekty klasy \texttt{entity::Entity} odnoszą się do siebie za pomocą \texttt{std::weak\_ptr}, przez co mogą sprawdzać czy nadal istnieją. Obiekt klasy \texttt{audio::AudioScene} może bez problemu usuwać dźwięki obiektów które już nie istnieją.
    \item \textbf{Wykorzystanie wielu narzędzi jakie daje nam \texttt{c++}} - Zostały użyte makra, czy to do wpisywania stałych, czy to do małych operacji, jednak zdarzyła się sytuacja gdzie makro utrudniło pracę nad projektem\footnote{Makro \texttt{\#define sqr(x) x*x} nie było dobrym pomysłem}.
    \item \textbf{System Entity Component} - zaimplementowano system w formie jaka została zamierzona, jednak jego forma nie jest optymalna. Wcześniejsze ustalenie i zaplanowanie systemu w formie pozwalającej na wielowątkowość było by lepsze.
    \item \textbf{Zapoznanie się z bibliotekami projektu FFmpeg} - Już przed
        rozpoczęciem projektu miałem duże doświadczenie z używaniem narzędzi
        \texttt{ffprobe} oraz \texttt{ffmpeg} w skryptach\footnote{W firmię, w
        której pracuje, rozwijam rozwiązania związane z transmisją wideo w
        sieci, właśnie z użyciem narzędzia \texttt{ffmpeg}}, ale dobre zaimplementowanie filtrowania audio okazało się być trudniejsze niż myślałem, dla tego na chwilę obecną filtrowanie przejęła na siebie biblioteka \texttt{OpenAL}.
\end{enumerate}

\subsection{Wnioski po zakończeniu prac}

Stworzony system obiektów i komponentów ma wiele wad:
\begin{itemize}
    \item Jednowątkowość, program potrafii zarządzać tysiącem obiektów bez
        większych problemów, jednak potrzebuje do tego 80\% czasu jednego wątku
        procesora \emph{ryzen 5 3600}.
    \item Klasa \texttt{entity::Entity} jest zbyt minimalistyczna, może należy
        rozważyć podejście nie mające jej wogóle\footnote{Podobnie do silinika
        \emph{Godot}.}.
    \item Błędne podejście z brakiem \textbf{prawdziwych} konstruktorów
        parametrycznych w klasach dziedziczących po
        \texttt{component::Component}. Wszystkie zamiast tego przyjmują zawsze
        jeden i ten sam argument.
    \item Brak sposobu na serializacje obiektów klasy \texttt{entity::EntitySystem}.
\end{itemize}

\subsubsection*{Pomysły na rozwiązanie tych problemów}

\begin{itemize}
    \item Podejście \emph{funkcyjne} zamiast \emph{obiektowego}, zamiast polimorficznych klas z wirtualnymi metodami struktury z wskaźnikami.
    \item Odnoszenie się komponentów między sobą przez \texttt{struct}'y stanu, gdzie podczas klatki: 
        \begin{itemize}
            \item Wszystkie \texttt{struct}'y stanu są kopiowane\footnote{lub tworzone są nowe}, tak by komponenty zmieniające swój stan podczas klatki nie wpływały na pozostałe.
            \item Komponenty zmieniają swój stan dowolnie, jednak jeśli chcą zmienić stan innego komponentu, to zamiast robić to bezpośrednio, to \emph{wrzucają} prośbę do kolejki.
            \item Komponenty zmieniają swój stan przetwarzając prośby z kolejki.
        \end{itemize}
        Taki \emph{,,protokół''} komunikacji między komponentami pozwoliłby na wielowątkowość w aplikacji.
\end{itemize}

\subsection{Znane błędy}

\begin{itemize}
    \item \texttt{libav} ma wycieki pamięci (około 198 kilobajtów dla wszystkich
        dźwięków)
    \item Przeciwnicy w większych grupach zaczynają wytwarzać swoje własne pole
        grawitacyjne, odziałujące na innych przeciwników.
\end{itemize}

\subsection{Ogólny opis architektury projektu}

Środkiem projektu jest klasa \texttt{entity::EntitySystem}, która jest rdzeniem projektu. Obiekty tej klasy mają za zadanie rysowanie odpowiednich komponentów na ekranie, zarządzanie samymi obiektami klasy \texttt{entity::Entity}, jak i kontrola nad fizyką oraz audio. Na rysunku \ref{rys:owners} przedstawiono które obiekty jakiej klasy posiadają jakie inne.

Dużą częścią projektu są obiekty klas pochodnych klasy \texttt{component::Component}, zawierają one właśnie metody które są uruchamiane w odpowiedniej kolejności co klatkę. Ich struktura jest stworzona tak że jeśli dodamy do danego obiektu \texttt{entity::Entity} dany komponent, to ten obiekt zyska jakieś możliwości, na przykład komponent \texttt{component::PhysicsBody} symuluje fizykę oraz kolizje.
\begin{figure}
    \resizebox{\textwidth}{!}{%
    \input{arch.tex}%
    }
    \centering
    \caption{Kto posiada kogo w aplikacji}\label{rys:owners}
\end{figure}

\subsection{Klasa \texttt{component::Component} i co się dzieje podczas klatki}

Polimorficzna klasa \texttt{component::Component} posiada metody wirtualne które są uruchamiane w odpowiedniej kolejności przez obiekty klasy \texttt{entity::EntitySystem}. Klasy pochodne przedstawiono na rysunku \ref{rys:component}.

Podczas trwania jednej klatki wywoływane są następujące metody:

\begin{itemize}
    \item \texttt{Awake()} - uruchamiane raz podczas dodawania entity z danym komponentem do obiekty klasy \texttt{entity::EntitySystem}
\end{itemize}

Reszta metod jest ściśle związana z tym co się dzieje podczas klatki:

\begin{enumerate}
    \item Na początku obiekty klasy \texttt{entity::Entity} są przydzielane do tablicy mieszającej\footnote{std::unordered\_map} w celu możliwości szybkiego wyszukiwania ich po lokalizacji.
    \item Jeżeli okno aplikacji jest aktywne, obiekt klasy \texttt{entity::InputHandler} wywołuje subskrybentów danych klawiszy.
    \item Następnie wywoływane są wszystkie metody \texttt{Update()} w komponentach.
    \item Potem wywoływane są metody \texttt{FixedUpdate()}, gdzie jako argument jest liczba sekund od ostatniej klatki.
    \item Wykonywana jest symulacja fizyki, tu są wywoływane metody \texttt{OnCollisionEnter()} oraz \texttt{OnCollisionLeave()} w komponentach klasy \texttt{component::PhysicsBody}, jako argument jest podawany wskaźnik do komponentu z którym zaszła kolizja.
    \item Po symulacji wywoływana jest metoda \texttt{LateFixedUpdate()}, gdzie powinno się usuwać obiekty.
    \item Wywoływana zostanie metoda \texttt{Update()} na obiekcie klasy \texttt{audio::AudioScene}, która porusza źródła dźwięku odpowiednio oraz usuwa te które albo skończyły już odtwarzać dźwięki lub ich właściciele nie istnieją.
    \item Na sam koniec klatki wywoływane zostają metody \texttt{Render} komponentów dziedziczących po klasie \texttt{component::Renderable}, zwracają one to, co ma zostać narysowane na ekranie.
\end{enumerate}


\begin{figure}
    \resizebox{\textwidth}{!}{%
    \input{renderable.tex}%
    }
    \centering
    \caption{Drzewo dziedziczenia komponentów}\label{rys:component}
\end{figure}

\section*{Spis assetów których sam nie zrobiłem:}

\begin{itemize}
    \item \texttt{./resources/antifachad.png} oraz \texttt{./resources/test.png} - Zbiór wojaków skompilowany przez \texttt{u/Reddit\_the\_xenomorph} dla społeczności \texttt{r/PoliticalCompassMemes}.
    \item \texttt{./resources/test.aac} - Anonimowy student \emph{University of Aberdeen} - dosłownie kolega kolegi.
    \item \texttt{./resources/shaders/explosion.frag} - ``Hyperspace explosion'' by Ben Wheatley - 2018 - Licencja MIT.
    
    Zmodyfikowany do współpracy z moim kodem.
\end{itemize}

\end{document}
