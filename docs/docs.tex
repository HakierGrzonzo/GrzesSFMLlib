% !TEX program = xelatex
%Wzór dokumentu
%tu zmień marginesy i rozmiar czcionki
\documentclass[a4paper,11pt]{article}
\usepackage{inputenc}[utf8]
\usepackage[margin=2.8cm]{geometry}
\usepackage[polish]{babel}

%Lepiej tego nie zmieniaj, jak co to dodawaj pakiety
\usepackage{titlesec}
\usepackage{titling}
\usepackage{fancyhdr}
\usepackage{mdframed}
\usepackage{graphicx}
\usepackage{amsmath}
\usepackage{amsfonts}
\usepackage{multicol}
\usepackage{multirow}
\usepackage{listings}
\usepackage{caption}
\usepackage{float}
\usepackage{pdfpages}
\usepackage{tikz}
	\usetikzlibrary{arrows}
	\usetikzlibrary{patterns}
	\usetikzlibrary{decorations.pathmorphing}
\usepackage{pgf}
\usepackage[section]{placeins}



%inny wygląd
%\usepackage{tgbonum}


\usepackage{hyperref}
\hypersetup{
    colorlinks=true,
    linkcolor=blue,
    filecolor=magenta,      
    urlcolor=cyan,
}

\urlstyle{same}
%Zmienne, zmień je!
\graphicspath{ {./ilustracje/} }
\title{Dokumentacja projektu GrzesSFMLlib}
\author{Grzegorz Koperwas}
\date{\today}

%lokalizacja polska (odkomentuj jak piszesz po polsku)

\usepackage{polski}
\usepackage[polish]{babel} 
\usepackage{indentfirst}
\usepackage{icomma} 

\brokenpenalty=1000
\clubpenalty=1000
\widowpenalty=1000    

%nie odkometowuj wszystkiego, użyj mózgu
%\renewcommand\thechapter{\arabic{chapter}.}
\renewcommand\thesection{\arabic{section}.}
\renewcommand\thesubsection{\arabic{section}.\arabic{subsection}.}
\renewcommand\thesubsubsection{\arabic{subsubsection}.}

%Makra

\newcommand{\obrazek}[2]{
\begin{figure}[h]
    \centering
    \includegraphics[scale=#1]{#2}
\end{figure}
}     

\newcommand{\stopnie}{\ensuremath{^{\circ}}}

\newcommand{\twierdzonko}[1]{
    \begin{center}
    \begin{mdframed}
    #1
    \end{mdframed}          
    \end{center}
} 

\newcommand{\dwanajeden}[2]{
\ensuremath \left( \begin{array}{c}
    #1\\
    #2
\end{array} \right)
}  

%Stopka i head (sekcja której nie powinno się zmieniać)
\pagestyle{fancy}
\fancyhead{}
\fancyfoot{}

%Zmieniaj od tego miejsca
\rfoot{\thepage}
\lfoot{}
\lhead{}
\rhead{}
\renewcommand{\headrulewidth}{1pt}
\renewcommand{\footrulewidth}{1pt}



\begin{document}
\begin{flushright}
    \LARGE{\thetitle}

    \Large{\theauthor}

    \thedate
\end{flushright}

\section{Instrukcje dotyczące kompilacji}

Projekt został wykonany w środowisku \texttt{meson}, zalecane jest użycie systemu linux i kompilacji krzyżowej w celu kompilacji projektu.

Projekt był testowany jedynie w architekturze x86\_64, i686\footnote{32 bitowe pentium III+} nie jest wspierane.


\subsection*{Potrzebne biblioteki i inne zalecane programy:}

\begin{itemize}
    \item Wymagane programy:
    \begin{itemize}
        \item gcc - kompilator
        \item meson - środowisko budujące
    \end{itemize}
    \item Zalecane programy:
    \begin{itemize}
        \item GNU make - pliki \texttt{makefile} w projekcie same nie budują jego, lecz jedynie wyzwalają komendy kompilacji i uruchomienia.
        \item fish - powłoka w której został napisany skrypt \texttt{refactor.sh}.
        \item ffmpeg - konwersja plików \texttt{.wav} do formatów skompresowanych.
        \item krita - edycja grafik (pliki \texttt{.kra} zawierają edytowalne wersje)
        \item sfxr - generowanie efektów dźwiękowych (pliki \texttt{.sfxr} zawierają edytowalne wersje)
        \item famitracker - generowanie innych efektów dźwiękowych (pliki \texttt{.ftm} zawierają edytowalne wersje)
        \item xelatex - dokumentacja.
    \end{itemize}
    \item Wymagane biblioteki:
    \begin{itemize}
        \item sfml
        \item box2d
        \item OpenAL
        \item \href{https://github.com/aaronmjacobs/Boxer}{boxer} - należy zainstalować bibliotekę samemu do \texttt{/lib64/libBoxer.a}
        \item libavcodec, libavutil, libavformat, libswresample - powinny być zainstalowane wraz z \texttt{ffmpeg}'iem
        \item GTK+ 3
    \end{itemize}
\end{itemize}

\subsection*{Kompilacja krzyżowa}

Kompilację krzyżową można zrealizować za pomocą narzędzi \texttt{mingw-w64-*}, jednak trzeba wtedy samodzielnie skompilować wszystkie biblioteki za pomocą tych narzędzi. \emph{Arch Linux} w repozytoriach społeczności posiada wersje wielu z tych programów i bibliotek pod mingw właśnie.

\section{Instrukcja obsługi}

\subsection*{Wymagania minimalne}

\begin{itemize}
    \item Karta graficzna z obsługą OpenGL 3.0 lub lepiej
    \item 64 bitowy system Gnu/Linux lub 64 bitowy system Windows
\end{itemize}

\subsection*{Sterowanie}

\begin{itemize}
    \item Poruszanie klawiszami \texttt{WASD}
    \item Strzelanie spacją
    \item Spowolnienie czasu - lewy ctrl
    \item Celowanie myszą
\end{itemize}

\section{Dokumentacja techniczna}

\subsection{Ogólny opis architektury projektu}

\begin{figure}[h]
    \resizebox{\textwidth}{!}{%
    \input{arch.tex}%
    }
    \centering
    \caption{Kto posiada kogo w aplikacji}
\end{figure}

\end{document}
